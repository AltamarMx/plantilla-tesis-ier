\chapter{Conclusiones}


En esta tesis se presentamos la simulaci'on num'erica usando el m'etodo de la ecuaci'on
de Boltzmann en redes (EBR) de la levitaci'on ac'ustica de una part'icula s'olida circular
bidimensional en una cavidad con un reflector plano y otra con un reflector redondeado 
y concluimos que el m'etodo de la EBR es capaz de 
simular correctamente el problema de levitaci'on ac'ustica de una part'icula s'olida.

En el cap'itulo 2 presentamos como llegar a la ecuaci'on de Boltzmann en redes
para la malla $D2Q9$  a partir de la ecuaci'on de transporte de Boltzmann. Tambi'en
obtuvimos las ecuaciones hidrodin'amicas para un fluido compresible a partir de la 
EBR usando una expansi'on de Chapman-Enskog. Presentamos brevemente
la t'ecnica desarrollada por Aidun {\it et al}~\cite{aidun98} para la interacci'on
de part'iculas s'olidas con el m'etodo de la  ecuaci'on de Boltzmann en redes. Adem'as,
describimos el rebote hacia atr'as y el rebote hacia atr'as a mitad del camino para
recrear la condici'on de no deslizamiento en las paredes as'i como la manera
de agregar cantidad de movimiento para despu'es simular una fuente ac'ustica. 
Ambas condiciones de frontera  las utilizamos para la simulaci'on de la levitaci'on ac'ustica.
La validaci'on del m'etodo de la EBR se present'o en el ap'endice A, donde estudiamos tres
problemas y comparamos los resultados de las simulaciones num'ericas con los reportados
en la literatura.


En el cap'itulo 3 presentamos los antecedentes de la levitaci'on ac'ustica y resultados
de las simulaciones.
Los trabajos de King~\cite{king34} y Gor'kov~\cite{gorkov62} fueron la base 
para el comienzo del desarrollo de la teor'ia de la fuerza ac'ustica actuando
sobre una esfera. Realizamos simulaciones num'ericas de la levitaci'on 
ac'ustica en dos cavidades, una con un reflector plano, y la otra con un reflector
redondeado optimizado para aumentar la fuerza del campo ac'ustico~\cite{xie01}.
Encontramos los dos primeros modos resonantes para la cavidad redondeada monitoreando
el valor de la velocidad m'axima. Todos los experimentos para ambas
cavidades los realizamos en el segundo modo resonante. Los valores de las frecuencias de resonancia 
encontrados difieren a los reportados por
Xie {\it et al}. Esta diferencia es porque las simulaciones num'ericas las realizamos
en dos dimensiones mientras que lo reportado por Xie {\it et al} es para una cavidad
en tres dimensiones. Comprobamos
que en ausencia de gravedad, las part'iculas se dirigen al nodo de presi'on en ambas
cavidades, donde la
sumatoria de fuerzas ac'usticas es cero. En el segundo conjunto de simulaciones
mantuvimos constante la cantidad de movimiento agregada por la fuente ac'ustica
y variamos la  relaci'on de densidades entre la part'icula y el fluido. 
La posici'on de equilibrio de la part'icula, para ambas cavidades,
se fue desplazando hacia abajo. En una peque~na zona, el desplazamiento de la 
posici'on de equilibri'o sigui'o una l'inea recta. En la cavidad plana, las oscilaciones en el eje vertical
disminuyen conforme la relaci'on de densidades aumenta, pero para la cavidad 
redondeada, hay una zona donde la oscilaci'on de la part'icula en el eje vertical aumenta.
Este comportamiento se debe a un desplazamiento en la frecuencia de resonancia debido
a la presencia de la part'icula s'olida~\cite{leung82}. En el tercer conjunto
de experimentos mantuvimos constante la densidad y variamos la cantidad de movimiento
agregada por la fuente ac'ustica. Conforme aumentamos la cantidad de movimiento, la
posici'on de la part'icula se desplaza hacia el nodo de presi'on m'as cercano. 
Existe un valor cr'itico en la cantidad de movimiento para la cavidad redondeada  a
partir del cual se forman dos nodos de presi'on a la misma altura. La aparici'on de los dos
nodos de presi'on a la misma altura, coincide con un movimiento irregular en el eje
vertical. La part'icula
oscila horizontalmente alrededor de estos dos nodos presentando un comportamiento 
complejo. Medimos el exponente de Lyapunov buscando que el sistema fuera sensible
a las condiciones iniciales, pero el valor del exponente fue tan bajo que no nos permite
decir de manera concluyente que el movimiento de la part'icula es ca'otico. Es posible
que la trayectoria de la part'icula presente un caos transiente, sin embargo es necesario
realizar m'as simulaciones para estudiar con detalle este fen'omeno.
Todas las trayectorias 
presentaron un movimiento oscilatorio con frecuencia de la fuente ac'ustica y sus arm'onicos.
Bajo ciertas condiciones, aparece una segunda frecuencia m'as baja que la de la fuente
ac'ustica tambi'en con sus arm'onicos.


El espacio de par'ametros que estudiamos en este trabajo se redujo a la variaci'on
de la cantidad de movimiento agregada por la fuente ac'ustica para un valor de la relaci'on
de densidades de la part'icula y el fluido. En otro conjunto de simulaciones num'ericas
estudiamos el comportamiento de la part'icula al variar la relaci'on de densidades y mantener
fija la cantidad de movimiento agregada por la fuente ac'ustica. Existen al menos dos par'ametros
importantes que son el radio de la part'icula s'olida y los par'ametros geom'etricos de la cavidad
redondeada, los cuales se mantuvieron fijos. Con los resultados que hemos presentado, podemos concluir
que el m'etodo de la ecuaci'on de Boltzmann en redes simula de manera correcta
la interacci'on de una part'icula s'olida en un fluido compresible.

Durante los estudios de doctorado se presentaron los siguientes trabajos en congresos:
\begin{itemize}
\item  	G. Barrios, R. Rechtman, Lattice Boltzmann equation for natural 
	convection inside a partially heated cavity, Statistical Mechanics, 
	Chaos and Condensed Matter, Roma, Italia, 22 al 24 de septiembre del 2004.

\item  	G. Barrios del Valle, R. Rechtman, J. Rojas, R. Tovar, Convecci'on 
	Natural en una Caja Parcialmente Calentada Usando el M'etodo de 
	la Ecuaci'on de Boltzmann en Redes, X Congreso Nacional 
	de la Divisi'on de Din'amica de Fluidos de la Sociedad Mexicana de
	F'isica, Hermosillo Son., 25 al 29 de octubre del 2004.

\item  G. Barrios, R. Rechtman, Levitaci'on Acu'stica usando el me'todo 
	de la ecuacio'n de Boltzmann en redes, XI Congreso de la Divisi'on 
	de Fluidos y Plasmas de la Sociedad Mexicana de F'isica, 
	Guadalajara, Jal., 17 al 20 de octubre del 2005.
\item  G. Barrios, R. Rechtman, Interacci'on de plumas t'ermicas 
	usando el m'etodo de la ecuaci'on de Boltzmann en redes, 
	XI Congreso de la Divisi'on de Fluidos y Plasmas de la Sociedad 
	Mexicana de F'isica, Guadalajara, Jal., 17 al 20 de octubre del 2005.
\item   G. Barrios, R. Rechtman, Formaci'on de Plumas T'ermicas 
	usando el M'etodo de la Ecuaci'on de Boltzmann en Redes, 
	XXI Congreso de la Divisi'on de Fluidos y  Plasmas 
	de la Sociedad Mexicana de F'isica, San Luis Potos'i,
	San Luis Potos'i, 19 de octubre del 2006

\item 	G. Barrios, R. Rechtman, Natural Convection using the Lattice Boltzmann
	Equation, 2006 APS Division of Fluid Dynamics 59th Annual Meeting, Tampa,
	Flo., E. U., 19 al 21 de noviembre del 2006.
\end{itemize}

Tambi'en se escribi'o un art'iculo y se envi'o para su publicaci'on al 
{\it Journal of Fluids Mechanics}, dicho art'iculo se encuentra
en el ap'endice~\ref{articulo}.

